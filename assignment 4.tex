\documentclass{article}                    % article class
 
\begin{document} % Begin document text
\begin{titlepage} % Suppresses headers and footers on the title page

	\centering % Centre everything on the title page
	
	\scshape % Use small caps for all text on the title page
	
	\vspace*{\baselineskip} % White space at the top of the page
	
	%------------------------------------------------
	%	Title
	%------------------------------------------------
	
		
	\vspace{0.80\baselineskip} % Whitespace above the title
	
	{\LARGE MAKERERE UNIVERSITY\\ COLLEGE OF COMPUTING AND INFORMATION SCIENCES\\ DEPARTMENT OF COMPUTER SCIENCE\\BIT2207: RESEARCH METHODOLOGY\\LECTURER: MR ERNEST MWEBAZE\\} % Title

{\LARGE TOPIC: A LITERATURE REVIEW ON CLOUD COMPUTING  ADOPTION IN ENTERPRISES\\}
	
	\vspace{8.00\baselineskip} % Whitespace below the title
	

	
	%------------------------------------------------
	%	Student
	%------------------------------------------------
	
	\vspace{0.5\baselineskip} % Whitespace before the student details
	
	{\scshape\Large NAME:OMODING JOHN\\REG. NO:16/U/11013/EVE\\STUDENT NO.216009128 \\} %student details
	
	\vspace{0.5\baselineskip} % Whitespace below the student details
\end{titlepage}
\newpage
\section{ABSTRACT}
Cloud computing is  the practice of using a network of remote servers hosted on the Internet to store, manage, and process data, rather than a local server or a personal computer.It  has received increasing interest from enterprises since its inception. With its innovative information technology (IT) services delivery model, cloud computing could add technical and strategic business value
to enterprises. However, cloud computing poses highly concerning internal (e.g., Top management and experience) and external issues (e.g., regulations and standards). This paper presents a systematic literature review to explore the current key issues related to cloud computing adoption.It's achieved by reviewing articles about cloud computing adoption. Using the grounded theory approach, articles are classified into eight main categories: internal, external, evaluation, proof of concept, adoption decision, implementationand integration, IT governance, and confirmation. Then, the eight categories are divided into two abstract categories: cloud computing adoption factors and processes, where the former affects the latter. The results of this review indicate that enterprises face serious issues before they decide to adopt cloud computing. Based on the findings, the paper provides a future information systems(IS) research agenda to explore the previously under-investigated areas regarding cloud computing adoption factors and processes. This paper calls for further theoretical, methodological, and empirical contributions to the research area of cloud computing adoption by enterprises.
\newline
{\textbf{Keywords}}: Cloud computing  adoption in enterprises.
\section{INTRODUCTION:}
Over the past decade, there has been a heightened interest in the adoption of cloud computing by enterprises.It  promises the potential to reshape the way enterprises acquire and manage their needs for computing resources efficiently and cost-effectively [1]. In line with the notion of shared services, cloud computing is considered an innovative model for IT service sourcing that generates value for the adopting enterprises [2].It enables enterprises to focus on their core business activities, and, thus, productivity is increased [3]. The adoption is growing rapidly due to the scalability, flexibility, agility, and simplicity it offers to enterprises [3–6]. A recent cross-sectional survey by [7] on the adoption rates of cloud computing by enterprises reported that 77 percent of large enterprises are adopting the cloud, whereas 73 percent  of small and medium-sized enterprises (SMEs) are adopting the cloud. It has been reported that 32 percent of large enterprises are testing the concept of cloud computing; 37 percent are already running applications on the cloud; and 17 percent  are using cloud infrastructure [7]. Contrarily, 19 percent of SMEs are testing the concept; 29 percent are running applications on the cloud; and 41 percent  are using cloud infrastructure[7].
\section{REFERENCES: }
1. Elragal, A., Haddara, M.: The Future of ERP Systems: Look Backward Before Moving Forward. Procedia Technol. 5, 21–30 (2012).
\newline
2. Su, N., Akkiraju, R., Nayak, N., Goodwin, R.: Shared Services Transformation : Conceptualization and Valuation from the Perspective of Real Options. Decis. Sci. 40, 381–402 (2009).
\newline
3. Garrison, G., Kim, S., Wakefield, R.L.: Success Factors for Deploying Cloud Computing. Commun. ACM. 55, 62–68 (2012).
\newline
4. Parakala, K., Udhas, P.: The Cloud Changing the Business Ecosystem,http://www.kpmg.com/IN/en/IssuesAndInsights ThoughtLeadership/The-Cloud-Changing-the-Business-Ecosystem.pdf, (2011).
\newline
5. Herhalt, J., Cochrane, K.: Exploring the Cloud: A Global Study of Governments’ Adoption of Cloud,http://www.kpmg.com/ES/es ActualidadyNovedades/ArticulosyPublicaciones/Documents/Exploring-the-Cloud.pdf, (2012).
\newline
6. Venters, W., Whitley, E. a: A Critical Review of Cloud Computing:Researching Desires and Realities. J. Inf. Technol. 27, 179–197
(2012).
\newline
7. RightScale: RightScale State of the Cloud Report,https://www.rightscale.com/pdf/rightscale-state-of-the-cloud-report-2013.pdf,
(2013).
\newline
8. Armbrust, M., Fox, A., Griffith, R., Joseph, A.D., Katz, R., Konwinski, A.,Lee, G., Patterson, D., Rabkin, A., Stoica, I., Zaharia, M.: Above the Clouds :A Berkeley View of Cloud Computing, (2009).
\newline
9. Sultan, N.A.: Reaching for the “Cloud”: How SMEs Can Manage. Int. J. Inf.Manage. 31, 272–278 (2011).
\end{document}